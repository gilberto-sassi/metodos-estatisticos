\documentclass[12pt, a4paper]{article}

%encoding
%--------------------------------------
\usepackage[T1]{fontenc}
\usepackage[utf8]{inputenc}
%--------------------------------------

%Portuguese-specific commands
%--------------------------------------
\usepackage[portuguese]{babel}
%--------------------------------------


%hyphenation
%Hyphenation rules
%--------------------------------------
\usepackage{hyphenat}
\hyphenation{
	ma-te-má-ti-ca 
	re-cu-pe-rar 
	in-for-ma-ções
	in-for-ma-ção
	a-fe-tam
	par-ti-cu-lar
	par-ti-cu-la-res
	u-ni-for-mi-da-de
	u-ni-for-mi-da-des
}
%--------------------------------------



\usepackage{amsmath}
\usepackage{amsfonts}
\usepackage{amssymb}
\usepackage{enumerate}
\usepackage{booktabs}
\usepackage{longtable}
\usepackage{graphicx}

\usepackage{geometry}
\geometry{margin=0.35in}


\begin{document}

\begin{center}
Universiadade Federal da Bahia\\
Instituto de Matemática e Estatística\\
Prof. Dr. Gilberto Pereira Sassi\\
\vspace{1cm}
Lista de exercícios -- probabilidade.
\vspace{1cm}
% $1^\circ$  Lista
\end{center}

\begin{enumerate}
    \item Uma moeda viciada de modo que a probabilidade de sair cara é 4 vezes maior que a de sair coroa. Para 2 lançamentos 
    independentes dessa moeda, determinar:
    \begin{enumerate}
     \item O espaço amostral.
     \item A probabilidade de sair somente uma cara.
     \item A probabilidade de sair pelo menos uma cara.
     \item A probabilidade de dois resultados iguais.
    \end{enumerate}


    \item Uma universidade tem 10 mil alunos dos quais 4 mil são considerados esportistas. Temos, ainda, que 500 alunos são do
    curso de biologia diurno, 700 de biologia noturno, 100 são esportistas e da biologia diurno e 200 são esportistas da biologia noturno. Um aluno é escolhido ao acaso e pergunta-se a probabilidade de:
    \begin{enumerate}
     \item Ser esportista.
     \item Ser esportista e aluno da biologia noturno.
     \item Não ser da biologia.
     \item Ser esportista ou aluno da biologia.
     \item Não ser esportista, nem aluno da biologia.
    \end{enumerate}
    
    \item Sejam $A$ e $B$ dois eventos em um dado espaço amostral, tais que $P(A)=0,2$, $P(B)=p$, $P(A \cup B)=0,5$ e
    $P(A \cap B)=0,1$. Determine o valor de $p$.
    
    \item Dois processadores tipo A e B são colocados em teste por 50 mil hora. A probabilidade de que um erro de cálculo aconteça em 
    um processador do tipo A é $\frac{1}{30}$, no tipo B $\frac{1}{80}$ e, em ambos, $\frac{1}{1000}$. Qual a probabilidade de que:
    \begin{enumerate}
     \item Pelo menos um dos processadores tenha apresentado erro?
     \item Nenhum processador tenha apresentado erro?
     \item Apenas o processador A tenha apresentado erro?
    \end{enumerate}

\item Das pacientes de uma Clínica de Ginecologia com idade acima de 40 anos, $60\%$ são ou foram casadas e $40\%$ são solteiras. Sendo solteira, a probabilidade de ter distúrbio hormonal no último ano é de $10\%$, enquanto que para as demais essa probabilidade aumenta para $30\%$. Pergunta-se:
\begin{enumerate}
	\item Qual a probabilidade de uma paciente escolhida ao acaso ter tido um distúrbio hormonal?
	\item Se a paciente sorteada tiver distúrbio hormonal, qual a probabilidade de ser solteira?
	\item Se escolhermos duas pacientes ao acaso e com reposição, qual é a probabilidade de pelo menos uma ter o distúrbio?
\end{enumerate}
\item Numa certa população, a probabilidade de gostar de teatro é $\frac{1}{3}$, enquanto que a de gostar de cinema é $\frac{1}{2}$. Determine a probabilidade de gostar de teatro e não de cinema, nos seguintes casos:
\begin{enumerate}
	\item Gostar de teatro e de cinema são eventos disjuntos;
	\item Gostar de teatro e de cinema são eventos independentes;
	\item Todos que gostam de teatro gostam de cinema;
	\item A probabilidade de gostar de teatro e cinema é $\frac{1}{8}$.
\end{enumerate}

\item Um médico desconfia que um paciente tem tumor no abdômen, pois isto ocorreu em $70\%$ dos casos similares que tratou. Se o paciente de fato tiver o tumor, o exame ultra-som detectará com probabilidade de 0,9. Entretanto, se ele não tiver o tumor, o exame pode, erroneamente, indicar que tem com probabilidade 0,1. Se o exame detectou um tumor, qual é a probabilidade do paciente tê-lo de fato?

\item Verifique se as seguintes afirmações são válidas:
\begin{enumerate}
	\item Se $P(A)=\frac{1}{3}$ e $P(B \mid A) = \frac{3}{5}$, então $A$ e $B$ não podem ser disjuntos.
	\item Se $P(A)=\frac{1}{2}$, $P(B \mid A)=1$ e $P(A \mid B)=\frac{1}{2}$, então $A$ não pode estar contido em $B$.
\end{enumerate}

\item Uma classe de estatística teve a seguinte distribuição de notas finais: 4 do sexo masculino e 6 do feminino foram reprovados, 8 do sexo masculino e 14 do feminino foram aprovados. Para um aluno sorteado dessa classe, denote por $M$ se o aluno for do sexo masculino e por $A$ se o aluno foi aprovado. Calcule:
\begin{enumerate}
	\item $P(A \cup M^c)$;
	\item $P(A^c \cap M^c)$;
	\item $P(A \mid M)$.
\end{enumerate}

\item Dois armários guardam as bolas de voleibol e basquete. O armário 1 tem 3 bolas de voleibol e 1 de basquete, enquanto 2 tem 3 bolas de voleibol e 2 de basquete. Escolhendo-se, ao acaso, um armário e, em sequida, um de suas bolsas, calcule a probabilidade dela ser:
\begin{enumerate}
	\item De voleibol, sabendo-se que o armário 1 foi escolhido;
	\item De basquete, sabendo-se que o armário 2 foir escolhido;
	\item De basquete.
\end{enumerate}

%\item Das pacientes de uma Clínica Ginecologia com idade acima de 40 anos, 60\% são ou foram casadas são casadas e 40\% são solteiras. Sendo solteira, a probabilidade de ter tido um distúrbio hormonal no último ano é de 10\%, enquanto que para as demais essa probabilidade aumenta para 30\%. Pergunta-se:
%\begin{enumerate}
%	\item Qual a probabilidade de uma paciente escolhida ao acaso ter tido um distúrbio hormonal?
%	\item Se a paciente sorteada fiver distúrbio hormonal, qual a probabilidade de ser solteira?
%	\item Se escolhermos duas pacientes ao acaso e com reprosição, qual a probabilidade de pelo menos uma ter o distúrbio?
%\end{enumerate}

\item Você entrega a seu amigo uma carta, destinada à sua namorada, para ser colocada no correio. Entretando, ele pode se esquecer com probabilidade 0,1. Se não se esquecer, a probabilidade de que o correio estravie a carta é de 0,1. Finalmente, se foi enviada pelo correio a probabilidade de que a namorada não a receba é 0,1. Sua namorada recebeu a carta, qual a probabilidade de seu amigo ter esquecido de colacá-la no correio.

%\item Um médico desconfia que um paciente tem tumor no abdômen, pois isto ocorreu em 70\% dos casos similares que tratou. Se o paciente de fato tiver o tumor, o exame ultra-som o decterá com probabilidade 0,9. Entretanto, se ele não tiver o tumor, o exame pode, erroneamente, indicar que tem com probabilidade 0,1. Se o exame detectou um tumor, qual é a probabilidade do paciente tê-lo de fato?

\item Três fábricas fornecem equipamentos de precisão para o laboratório de química de uma universidade. Apesar de serem aparelhos de precisão, existe uma pequena chance de subestimação ou superestimação das medidas efetuadas. A tabela~\ref{tab:fab} a seguir apresenta o comportamento do equipamento produzido em cada fábrica:
\begin{table}[htbp]
	\centering
	\caption{Probabilidades para cada fábrica.}
	\label{tab:fab}
	\begin{tabular}{l|ccc}
		\toprule[0.05cm]
		& \multicolumn{3}{|c}{Probabilidade}\\ \cmidrule[0.025cm]{2-4}
		& Subestima & Exata & Superestima \\ \midrule[0.05cm]
		Fábrica I & 0,01 & 0,98 & 0,01 \\
		Fábrica II & 0,005 & 0,98 & 0,015 \\
		Fábrica III & 0,00 & 0,99 & 0,01 \\ \bottomrule[0.05cm]
	\end{tabular}
\end{table}
As fábricas I, II, III fornecem, respectivamente, 20\%, 30\% e 50\% dos aparelhos utilizados. Escolhemos, ao acaso, um desses aparelhos e perguntamos qual é a probabilidade de:
\begin{enumerate}
	\item Haver superestimação de medidas?
	\item Não haver subestimação de medidas?
	\item Dando medidas exatas, ter sido fabricado em III?
	\item Ter sido produzido por I, dado que não subestima as medidas?
\end{enumerate}

\item A aplicação de fundo anticorrosivo em chapas de aço de $1m^2$ é feita mecanicamente e pode produzir defeitos (pequenas bolhas na pintura), de acordo com uma variável aleatória Poisson de parâmetro $\lambda=1$ por $m^2$. Uma chapa é sorteada ao acaso para ser inspecionada, pergunta-se a probabilidade de:
\begin{enumerate}
	\item Encontrarmos pelo menos um defeito;
	\item No máximo 2 defeitos serem encontrados;
	\item Encontrarmos de 2 a 4 defeitos;
	\item Não mais de 1 defeito ser encontrado.
\end{enumerate}

\item Uma variável aleatória discreta $X$ tem a seguinte função de distribuição acumulada:
\begin{align*}
F(x) = \begin{cases}
0 & \mbox{ se } x < -1,\\
0,2 & \mbox{ se } -1 \leq x < 2,\\
0,5 & \mbox{ se } 2 \leq x < 5,\\
0,7 & \mbox{ se } 5 \leq x < 6,\\
0,9 & \mbox{ se } 6 \leq x < 15,\\
1 & \mbox{ se } x \geq 15.
\end{cases}
\end{align*}
Determine:
\begin{itemize}
	\item A função de probabilidade de $X$;
	\item $P(X \leq -2)$;
	\item $P(X < 2)$;
	\item $P(3 \leq X \leq 12)$ ;
	\item $P(X  > 14)$.
\end{itemize}

\item Um time paulista de futebol tem probabilidade 0,92 de vitória sempre que joga. Se o time atuar 4 vezes, determine a probabilidade de que vença:
\begin{enumerate}
	\item Todas as 4 partidas;
	\item Exatamente 2 partidas;
	\item Pelo menos uma partida;
	\item No máximo 3 partidas.
\end{enumerate}

\item Um vacina contra a gripe é eficiente em 70\% dos casos. Sorteamos, ao acaso, 20 dos pacientes vacinados e pergunta-se a probabilidade de obter:
\begin{enumerate}
	\item Pelo menos 18 imunizados;
	\item No máximo 4 imunizados;
	\item Não mais do que 3 não imunizados.
\end{enumerate}

\item A resistência (em toneladas) de vigas de concreto produzidas por uma empresa, comporta-se conforme a função de probabilidade tabela~\ref{tab:res}.
\begin{table}[!htbp]
	\centering
	\caption{Função de probabilidade}
	\label{tab:res}
	\begin{tabular}{l|ccccc}
		\toprule[0.05cm]
		Resistência & 2 & 3 & 4 & 5 & 6 \\ \midrule[0.05cm]
		$p_i$ & 0,1 & 0,1 & 0,4 & 0,2 & 0,2 \\ \bottomrule[0.05cm]
	\end{tabular}
\end{table}
Admita que essas vigas são aprovadas para uso em contratações se suportam pelo menos 3 toneladas. De uma grande lote fabricado pela empresa, escolhemos 15 vigas ao acaso. Qual será a probabilidade de:
\begin{enumerate}
	\item Todas serem aptas para construções?
	\item No mínimo 13 serem aptas?
\end{enumerate}

\item As pacientes diagnosticadas com câncer de mama precocemente têm 80\% de probabilidade de serem completamente curadas. Para um grupo de 12 pacientes nessas condições, calcule a probabilidade de:
\begin{enumerate}
	\item Oito ficarem completamente curadas;
	\item Não serem curadas de 3 a 5 pacientes;
	\item Não mais de 2 permancerem com a doença.
\end{enumerate}

\item Uma indústria de tintas recebe pedidos de seus vendedores através de fax, telefone e Internet. O número de pedidos que chegam por qualquer meio (no horário comercial) é uma variável aleatória discreta com distribuição Poisson com taxa de 5 pedidos por hora.
\begin{enumerate}
	\item Calcule a probabilidade de mais de 2 pedidos por hora;
	\item Em um dia de trabalho (8 horas), qual seria a probabilidade de haver 50 pedidos?
	\item Não haver nenhum pedido, em um dia de trabalho, é um evento raro?
\end{enumerate}

\item No estudo do desempenho de uma central de computação, o acesso à Unidade Central de Processamento é assumido ser Poisson com 4 requisições por segundo. Essas requisições podem ser de várias naturezas tais como: imprimir um arquivo, efetuar um certo cálculo ou enviar uma mensagem pela Internet, entre outras. Escolhendo-se ao acaso um intervalo de 1 segundo, qual é a probabilidade de haver mais de 2 acessos à CPU? E do número de acessos não ultrapassar 5?

\item O tempo adequado de troca do conjunto de amortecedores de certa marca em automóveis, sujeitos a uso contínuo e severo, pode ser considerado como uma variável contínua, medida em anos. Suponha que a função densidade é dada pela seguinte expressão:
\begin{align*}
f(x) = \begin{cases}
\dfrac{x}{4} & 0 \leq x \leq 2;\\
\dfrac{1}{8} & 2 < x \leq 6;\\
0 & \mbox{caso contrário}.
\end{cases}
\end{align*}
\begin{enumerate}
	\item Verifique que a função acima é, de fato, uma densidade;
	\item Qual é a probabilidae de um automóvel, sujeito às condições descritas acima, necessitar de troca de amortecedores antes de 1 ano de uso? E entre 1 e 3 anos?
\end{enumerate}


\item Suponha que o peso de recém-nascidos (em kg) pode ser considerado uma variável aleatória com a seguinte densidade:
\begin{align*}
f(x) = \begin{cases}
\dfrac{x+1}{10} & 0 \leq x \leq 2;\\
\dfrac{-3\cdot x + 18}{40} & 2 < x \leq 6;\\
0 & \mbox{caso contrário}.
\end{cases}
\end{align*}
Qual a probabilidade de, escolhendo ao acaso uma criança, ela ter peso:
\begin{enumerate}
	\item Inferior a 3 kg?
	\item Entre 1 e 4 kg?
	\item Pelo menos 3 kg?
\end{enumerate}

\item O tempo de corrosão, em anos, de uma certa peça metálica é uma variável aleatória contínua com função de  densidade:
\begin{align*}
f(x) = \begin{cases}
a \cdot x & 0 \leq x \leq 1;\\
a  & 1 \leq x \leq 2;\\
-a\cdot x + 3 \cdot a & 2 < x \leq 3;\\
0 & \mbox{caso contrário}.
\end{cases}
\end{align*}
\begin{enumerate}
	\item Calcule a constante $a$;
	\item Uma peça é considerada como tendo boa resistência à corrosão se dura mais que 1,5 anos. Em um lote de 3 peças, qual a probabilidade de termos exatamente 1 delas com boa resistência?
\end{enumerate}

\item Dois amigos planejam um encontro entre 20 e 21 horas. Um deles é pontual e pretende chegar às 20:30 horas e esperar exatos 15 minutos. O outro é mais imprevisível e poderá chegar em qualquer momento do intervalo incialmente previsto, saindo imediatamente se não encontrar o amigo. Qual é a probabilidade de eles se encontrarem? Qual é a probabilidade de eles não se encontrarem por um lapso de no máximo 5 minutos?

\item O tempo, em minutos, de utilização de uma caixa eletrônico por clientes de um certo banco, foi modelado por uma variável aleatória contínua $T$ com modelo exponencial com taxa de decaimento $\alpha=3$. Determine:
\begin{enumerate}
	\item $P(T < 1)$;
	\item $P(T > 1 \mid T \leq 2)$;
	\item Um número $a$ tal que $P(T \leq a) = 0,4$.
\end{enumerate}

\item Seja $X \sim N(5,4)$. Determine:
\begin{enumerate}
	\item $P(X \leq 6)$;
	\item $P(7 < X < 8)$;
	\item $P(2 \leq X < 5)$;
	\item $P(-1 \leq X \leq 2)$;
	\item $P(X \leq -1)$;
	\item $P(-2 \leq X \leq -1)$.
\end{enumerate}

\item Um teste de aptidão feito por pilotos de aeronaves em treinamento inicial requer que uma série de operações seja realizada em um rápida sucessão. Suponha que o tempo necessário para completar o teste seja distribuído de acordo com uma Normal com média 90 minutos e desvio padrão de 20 minutos.
\begin{enumerate}
	\item Para passar no teste, o candidato deve completá-lo em menos de 80 minutos. Se 65 candidatos fazem o teste, quantos são esperados passar?
	\item Se os 5\% melhores candidatos são alocados para aeronaves maiores, quão rápido deve ser o candidato para que obtenha essa posição?
\end{enumerate}

\item Com base em experiências anteriores, a Companhia Telefônica sabe que 10\% das contas dos seus clientes em uma comunidade são pagas com atraso. Para os itens abaixo, use a aproximação da variável aleatória contínua Normal.
\begin{enumerate}
	\item Se 20 contas são enviadas em um dia de pela Companhia Telefônica, qual é a probabilidade de que menos do que 3 sejam pagas com atraso?
	\item Se 150 contas são enviadas mensalmente para a comunidade, encontre a probabilidade de que 17 ou mais sejam pagas com atraso.
\end{enumerate}

\item A durabilidade de um tipo de pneu da marca \textit{Rodabem} é descrita por uma variável aleatória contínua Normal de média 60.000 km e desvio padrão 8.300 km. 
\begin{enumerate}
	\item Se a \textit{Rodabem} garante os pneus pelos primeiros 48.000 km, qual a proporção de pneus que deverão ser trocados pela garantia?
	\item Qual deveria ser a garantia com a proporção do item (a), se a garantia fosse para os primeiros 45.000 km?
	\item Qual deveria ser a garantia (em km) de tal forma a assegurar que o fabricante trocaria sob garantia no máximo 2\% dos pneus?
	\item Se você comprar 4 pneus \textit{Rodabem}, qual será a probabilidade de que você utilizaria a garantia (45.000 km ) para trocar um ou mais desses pneus?
\end{enumerate}

\end{enumerate}

\end{document}
